\documentclass[conference]{IEEEtran}
\IEEEoverridecommandlockouts
\usepackage{matlab-prettifier}
\usepackage{cite}
\usepackage{amsmath,amssymb,amsfonts}
\usepackage{algorithmic}
\usepackage{graphicx}
\usepackage{textcomp}
\usepackage{xcolor}
\usepackage{float}
\def\BibTeX{{\rm B\kern-.05em{\sc i\kern-.025em b}\kern-.08em
    T\kern-.1667em\lower.7ex\hbox{E}\kern-.125emX}}

\begin{document}

\title{Project 4 : Order Statistics}

\author{\IEEEauthorblockN{Owen Sowatzke}
\IEEEauthorblockA{\textit{Electrical Engineering Department} \\
\textit{University of Arizona}\\
Tucson, USA \\
osowatzke@arizona.edu}}
\maketitle

\begin{abstract}
Assume an FFT is used to detect a tone in the presence of noise. If the absolute value of the spectrum is taken and compared to a threshold, what is the probability that the largest noise spectrum sample will trigger the detection logic? To solve this problem, the probability density function (pdf) of the largest noise spectrum sample is required. In this paper, a set of iid random variables $\{X_1,...,X_N\}$ will be considered. %This set of random variable could be the absolute value of the noise FFT. 
This set of random variables will be sorted in increasing order to form the ordered statistics, denoted as $\{X_{(1)},...,X_{(N)}\}$. Then, the pdf of the first-order statistic, $f_{X_{(1)}}(x)$, and the pdf of the Nth-order statistic, $f_{X_{(N)}}(x)$, will be derived. Finally, the pdf of the overall order statistics will be derived ($f_{X_{(n)}}(x)$ for $n\in [1, N]$). The results derived in this paper can not only be applied to the FFT problem, but to many other engineering problems.
%Consider the set of N iid random variables $\{X_1,...,X_N\}$. The set of random variables can be sorted in ascending order to form a ordered list of random variables $\{X_{(1)},...,X_{(N)}\}$. The n-th ordered statistic is the random variable $X_{(n)}$. This paper derives the pdf of the first-order statistic $X_{(1)}$ and the pdf of the Nth-order statistic $X_{(N)}$. Then, the results are generalized to form the overall order statistics (the pdf of the nth-order statistic $X_{(n)}$ where $N\in [1, N]$).
\end{abstract}

\begin{IEEEkeywords}
Random Variables, Order Statistics, Probability Density Function, Cumulative Distribution Function
\end{IEEEkeywords}

\section{Introduction}
Consider N random variables $\{X_1,...,X_N\}$ that are iid with pdf $f_X(x)$. The set of random variables are sorted in ascending order to produce the ordered statistics, denoted as $\{X_{(1)},...,X_{(N)}\}$. The first order statistic is defined as the following:
\begin{equation}
X_{(1)}\triangleq\text{min}\{X_1,...,X_N\}
\end{equation}
Similarly, the Nth-order statistic is defined as the following:
\begin{equation}
X_{(N)}\triangleq\text{max}\{X_1,...,X_N\}
\end{equation}
In this document, the pdf of the first-order statistic $f_{X_{(1)}}(x)$ and the pdf of the Nth-order statistic $f_{X_{(N)}}(x)$ are derived. Finally, the pdf for the overall order statistics is derived (i.e. $f_{X_{(n)}}(x)$ for any $n\in[1,N]$).
%(the pdf of $X_{(n)} \quad \forall n \in[1, N]$) are derived.
% $\triangleq\text{min}\{X_1,...,X_N\}$


%The goal in this section is to derive pdfs for the minimum $X_{(1)}=\text{min}(X_1,...,X_N)$ (called the first-order statistic), the maximum $X_{(1)}=\text{max}(X_1,...,X_N)$ (called the N-th order statistic), as well as the overall order statistics. 
%Sort these variables in ascending order. If the random variables are sorted in increasing order, the   

%Suppose N random variables are drawn from the distribution of the random variable X {
%
%The order statistics of a random variable are defined as the pdf.
\section{First-Order Statistic}

\label{First-Order Statistic Section}

%% Orignal Code
%The cdf of the first-order statistic is given by
%\begin{equation}
%F_{X_{(1)}}(x) = P(\{X_{(1)} \leq x\})
%\end{equation}
%Note that there are many ways for $X_{(1)}$ to be less than or equal to $x$. To simplify this problem, $F_{X_{(1)}}(x)$ can be expressed in the following form:
%\begin{equation}
%F_{X_{(1)}}(x) = 1 - P(\{X_{(1)} > x\})
%\end{equation}
%For $X_{(1)}$ to be greater than $x$, each $X$ ($X_1,...,X_N$) must be greater than $x$.
%\begin{equation}
%F_{X_{(1)}}(x) = 1 - P(\{X_1 > x \} \cap \cdots \cap \{X_N > x \})
%\end{equation}
%Because $X_1,...,X_N$ are independent with the same distribution, the cdf can be expressed in the following form:
%\begin{equation}
%F_{X_{(1)}}(x) = 1 - [P(\{X > x\})]^N
%\end{equation}
%$P(\{X > x\})$ can be rewritten in terms of $F_X(x)$. This provides the following result:
%\begin{equation}
%\label{1st order cdf}
%F_{X_{(1)}}(x) = 1 - [1 - F_X(x)]^N
%\end{equation}
%The pdf of the first order statistic is the derivative of the cdf given in equation \eqref{1st order cdf}.
%\begin{equation}
%F_{X_{(1)}}(x) = -N[1 - F_X(x)]^{N-1}(-f_X(x))
%\end{equation}
%After simplifying the derivative, the pdf of the first order statistic can be written as follows:
%\begin{equation}
%f_{X_{(1)}}(x) = N[1 - F_X(x)]^{N-1}f_X(x)
%\end{equation}


%% New Code
The cumulative distribution function (cdf) of the first-order statistic can be expressed as follows:
\begin{equation}
F_{X_{(1)}}(x) = P(\{X_{(1)} \leq x\})
\end{equation}
%  is smaller than or equal to $x$,
If $X_{(1)} \leq x$, no direct conclusions can be drawn about %not much can be said about 
$X_{(2)},...,X_{(N)}$. The set of random variables $X_{(2)},...,X_{(N)}$ can be larger than or smaller than $x$ and still provide $X_{(1)} \leq x$. Note that the cdf of the first-order statistic can also be expressed in the following form:
\begin{equation}
F_{X_{(1)}}(x) = 1 - P(\{X_{(1)} > x\})
\end{equation}
If $X_{(1)} > x$, $X_{(2)},...,X_{(N)}$ must be greater than $x$. Therefore, $X_{(1)} > x$ is equivalent to each of the random variables $X_1,...,X_N$ being greater than $x$. 
%This can be expressed in the following form:
\begin{equation}
F_{X_{(1)}}(x) = 1 - P\left(\bigcap_{i=1}^N\{X_i > x\}\right)
\end{equation}
Because the random variables $X_1,...,X_N$ are independent, the cdf of the first-order statistic can be written as follows:
\begin{equation}
F_{X_{(1)}}(x) = 1 - \prod_{i=1}^N P(\{X_i > x\})
\end{equation}
This equation can be rewritten in terms of the distribution functions $F_{X_1}(x),...,F_{X_N}(x)$ as follows:
\begin{equation}
\begin{gathered}
F_{X_{(1)}}(x) = 1 - \prod_{i=1}^N (1-P(\{X_i \leq x\}))\\
= 1 - \prod_{i=1}^N (1-F_{X_i}(x))
\end{gathered}
\end{equation}
Each of the random variables $X_1,...,X_N$ has the same distribution.
\begin{equation}
F_{X_1}(x)=\cdots=F_{X_N}(x)=F_X(x)
\end{equation}
Therefore, the cdf of the first-order statistic can be expressed as follows: 
\begin{equation}
F_{X_{(1)}}(x) = 1 - [1 - F_X(x)]^N
\end{equation}
The probability density function (pdf) of the first-order statistic is the derivative of the cdf with respect to $x$.
\begin{equation}
f_{X_{(1)}}(x) = \frac{d}{dx}\left\{F_{X_{(1)}}(x)\right\}
\end{equation}
Solving for the derivative, the following equation results:
\begin{equation}
f_{X_{(1)}}(x) = -N[1 - F_X(x)]^{N-1}(-f_X(x))
\end{equation}
Finally, after simplifying, the pdf of the first-order statistic can be expressed as follows:
\begin{equation}
f_{X_{(1)}}(x) = N[1 - F_X(x)]^{N-1}f_X(x)
\end{equation}
\section{Nth-Order Statistic}
\label{Nth-Order Statistic Section}

% Original Code
%The cdf of the Nth-order statistic is given by
%\begin{equation}
%F_{X_{(N)}}(x) = P(\{X_{(N)} \leq x\})
%\end{equation}
%For $X_{(N)}$ to be less than or equal to $x$, each $X$ ($X_1,...,X_N$) must be less than or equal to $x$.
%\begin{equation}
%F_{X_{(N)}}(x) = P(\{X_1 \leq x \} \cap \cdots \cap \{X_N \leq x \})
%\end{equation}
%Because $X_1,...,X_N$ are independent with the same distribution, the cdf can be expressed in the following form:
%\begin{equation}
%F_{X_{(N)}}(x) = [P(\{X \leq x \}]^N
%\end{equation}
%Substituting $F_X(x)$ for $P(\{X \leq x \}$, the cdf is given by the following:
%\begin{equation}
%\label{Nth order cdf}
%F_{X_{(N)}}(x) = [F_X(x)]^N
%\end{equation}
%The pdf of the Nth-order statistic is the derivative of the cdf given in equation \eqref{Nth order cdf}. After taking the derivative, the following result ensues.
%\begin{equation}
%f_{X_{(N)}}(x) = N[F_X(x)]^{N-1}f_X(x)
%\end{equation}

% New Code
The cdf of the Nth-order statistic function is given by the following equation:
\begin{equation}
F_{X_{(N)}}(x) = P(\{X_{(N)} \leq x\})
\end{equation}
If $X_{(N)} \leq x$, $X_{(1)},...,X_{(N-1)}$ must be less than or equal to $x$. Therefore, $X_{(N)} \leq x$ is equivalent to each of the random variables $X_1,...,X_N$ being less than or equal to $x$.
\begin{equation}
F_{X_{(N)}}(x) = P\left(\bigcap_{i=1}^N\{X_i \leq x\}\right)
\end{equation}
Because the random variables $X_1,...,X_N$ are independent, the cdf of the Nth-order statistic can be written as follows:
\begin{equation}
F_{X_{(N)}}(x) = \prod_{i=1}^N P(\{X_i \leq x\})
\end{equation}
This equation can be rewritten in terms of the distribution functions $F_{X_1}(x),...,F_{X_N}(x)$ as follows:
\begin{equation}
F_{X_{(N)}}(x) = \prod_{i=1}^N F_{X_i}(x)
\end{equation}
Because each of the random variables has the same distribution function, the cdf of Nth-order statistic function, can be expressed as follows:
\begin{equation}
F_{X_{(N)}}(x) = [F_X(x)]^N
\end{equation}
The pdf of the Nth-order statistic can be found by differentiating the cdf with respect to x. Solving for the derivative, the following equation results:
\begin{equation}
f_{X_{(N)}}(x) = N[F_X(x)]^{N-1}f_X(x)
\end{equation}
\section{Overall order Statistics}
The results from Section \ref{First-Order Statistic Section} and \ref{Nth-Order Statistic Section} can be generalized, to provide the pdf for the nth-order statistic where $n\in[1,N]$. The pdf of the nth-order statistic function can be derived using the cdf of the nth-order statistic function. The cdf is given by the following equation:
\begin{equation}
F_{X_{(n)}}(x) = P(\{X_{(n)} \leq x\})
\end{equation}
Define the events $A_1$ and $A_2$ as follows:
\begin{equation}
A_1 = \{X \leq x\}
\end{equation}
\begin{equation}
A_2 = \{X > x\}
\end{equation}
The probability that there are exactly $k$ occurrences of $A_1$ and $N-k$ occurrences of $A_2$ is given by
\begin{equation}
P_n(k) = \binom{N}{k}[P(A_1)]^k[P(A_2)]^{N-k}
\end{equation}
For $X_{(n)} \leq x$, there must be at least n occurrences of event $A_1$. The number of occurrences of event $A_2$ can range from $N-n$ downto $0$. Since these events are disjoint,  the cdf of the nth-order statistic can be expressed as a sum of probabilities.
\begin{equation}
\label{prob sum}
F_{X_{(n)}}(x) = \sum_{k=n}^{N}\binom{N}{k}[P(A_1)]^k[P(A_2)]^{N-k}
\end{equation}
The probability of events $A_1$ and $A_2$ are given by the following equations:
\begin{equation}
P(A_1) = P(\{X \leq x\}) = F_X(x)
\end{equation}
\begin{equation}
P(A_2) = P(\{X > x\}) = 1 - F_X(x)
\end{equation}
Substituting these probabilities into equation \eqref{prob sum}, the cdf of the nth-order statistic can be written as follows:
\begin{equation}
F_{X_{(n)}}(x) = \sum_{k=n}^{N}\binom{N}{k}[F_X(x)]^k[1-F_X(x)]^{N-k}
\end{equation}
The pdf of the nth-order statistic can be found by differentiating the cdf with respect to $x$. The resulting pdf is given by the following equation:
\begin{equation}
\begin{gathered}
f_{X_{(n)}}(x) = \sum_{k=n}^{N}\binom{N}{k}k[F_X(x)]^{k-1}f_X(x)[1-F_X(x)]^{N-k}\\
 + \sum_{k=n}^{N}\binom{N}{k}[F_X(x)]^{k}(N-k)[1-F_X(x)]^{N-k-1}(-f_X(x))
\end{gathered}
\end{equation}
The second summation is 0 when $k=N$. After removing the zero term and simplifying the equation, the pdf of the nth-order statistic can written as follows:
\begin{equation}
\begin{gathered}
f_{X_{(n)}}(x) = f_X(x)\sum_{k=n}^{N}\binom{N}{k}k[F_X(x)]^{k-1}[1-F_X(x)]^{N-k}\\
- f_X(x)\sum_{k=n}^{N-1}\binom{N}{k}(N-k)[F_X(x)]^{k}[1-F_X(x)]^{N-k-1}
\end{gathered}
\end{equation}
If the binomial coefficients are expanded, the resulting pdf can be further reduced:
\begin{equation}
\begin{gathered}
f_{X_{(n)}}(x) = f_X(x)\sum_{k=n}^{N}\frac{N!}{k!(N-k)!}k[F_X(x)]^{k-1}\\
\cdot[1-F_X(x)]^{N-k} - f_X(x)\sum_{k=n}^{N-1}\frac{N!}{k!(N-k)!}\\
\cdot(N-k)[F_X(x)]^{k}[1-F_X(x)]^{N-k-1}
\end{gathered}
\end{equation}
After canceling terms in the numerator of the expanded binomial coefficients, the following equation results:
\begin{equation}
\begin{gathered}
f_{X_{(n)}}(x) = f_X(x)\sum_{k=n}^{N}\frac{N!}{(k-1)!(N-k)!}[F_X(x)]^{k-1}\\
\cdot[1-F_X(x)]^{N-k} - f_X(x)\sum_{k=n}^{N-1}\frac{N!}{k!(N-k-1)!}\\
\cdot[F_X(x)]^{k}[1-F_X(x)]^{N-k-1}
\end{gathered}
\end{equation}
After rearranging terms, the equation can be reexpressed as follows:
\begin{equation}
\begin{gathered}
f_{X_{(n)}}(x) = f_X(x)\sum_{k=n}^{N}\frac{N(N-1)!}{(k-1)!(N-k)!}[F_X(x)]^{k-1}\\
\cdot[1-F_X(x)]^{N-k} - f_X(x)\sum_{k=n}^{N-1}\frac{N(N-1)!}{k!(N-k-1)!}\\
\cdot[F_X(x)]^{k}[1-F_X(x)]^{N-k-1}
\end{gathered}
\end{equation}
This equation can be rewritten in terms of binomial coefficients.
\begin{equation}
\begin{gathered}
f_{X_{(n)}}(x) = Nf_X(x)\sum_{k=n}^{N}\binom{N-1}{k-1}[F_X(x)]^{k-1}\\
\cdot[1-F_X(x)]^{N-k} - Nf_X(x)\sum_{k=n}^{N-1}\binom{N-1}{k}\\
\cdot[F_X(x)]^{k}[1-F_X(x)]^{N-k-1}
\end{gathered}
\end{equation}
If $m$ is substituted for $k-1$ in the first summation, the equation can be rewritten as:
\begin{equation}
\begin{gathered}
f_{X_{(n)}}(x) = Nf_X(x)\sum_{m=n-1}^{N-1}\binom{N-1}{m}[F_X(x)]^{m}\\
\cdot[1-F_X(x)]^{N-m-1} - Nf_X(x)\sum_{k=n}^{N-1}\binom{N-1}{k}\\
\cdot[F_X(x)]^{k}[1-F_X(x)]^{N-k-1}
\end{gathered}
\end{equation}
After the second sum is subtracted from the first, all that remains is the first term of the first summation.
\begin{equation}
f_{X_{(n)}}(x) = Nf_X(x)\binom{N-1}{n-1}[F_X(x)]^{n-1}[1-F_X(x)]^{N-n}
\end{equation}
Expanding the remaining binomial coefficient, the pdf of the nth-order statistic can be expressed as follows:
\begin{equation}
\begin{gathered}
f_{X_{(n)}}(x) = \frac{N!}{(n-1)!(N-n)!}\\
\cdot[F_X(x)]^{n-1}[1-F_X(x)]^{N-n}f_X(x)
\end{gathered}
\end{equation}
Substituting $n=1$ and $n=N$, the pdf of the first-order and Nth-order statistics can be confirmed. 
%% Old Material
%The results from Section \ref{First-Order Statistic Section} and \ref{Nth-Order Statistic Section} can be generalized, to provide the pdf for nth-order statistic where $n\in[1,N]$. 
%\begin{figure}[H]
%\centerline{\includegraphics[width=0.5\textwidth]{prob_pdf.png}}
%\caption{Probability that x lies in interval $(x, x+dx]$.}
%\label{Probability of Interval}
%\end{figure}
%\noindent
%Referring to Fig. \ref{Probability of Interval}, the pdf of the n-th order statistic can be expressed in terms of a probability as follows: 
%\begin{equation}
%f_{X_{(n)}}(x)dx = P(\{x < X_{(n)} \leq x+dx\})
%\end{equation}
%For $X_{(n)}$ to lie in interval $(x,x+dx]$, n-1 $X$'s must lie in the interval $(-\infty,x]$, 1 $X$ must lie in the interval $(x,x+dx]$, and N-n $X$'s must lie in the interval $(x+dx,\infty)$. 
%\par
%Let $A_1,A_2,\text{ and }A_3$ be the events that $X$ lies in the intervals $(-\infty,x],(x,x+dx],\text{ and }(x+dx,\infty)$ respectively.
%\begin{equation}
%A_1 = \{X \leq x\}
%\end{equation}
%\begin{equation}
%A_2 = \{x < X \leq x + dx\}
%\end{equation}
%\begin{equation}
%A_3 = \{X > x + dx\}
%\end{equation}
%The probability that $X_{(n)}$ lies in the interval $(x,x+dx]$, can be viewed as taking N samples from the distribution $F_X(x)$. The probability of interest is then the probability of n-1 occurences of $A_1$, 1 occurence of $A_2$, and N-n occurences of $A_3$.
%\begin{equation}
%\begin{gathered}
%P\{x < X_{(n)} \leq x + dx\}=\\
%\frac{N!}{(n-1)!1!(N-n)!}(P(A_1))^{n-1}P(A_2)(P(A_3))^{N-n}
%\end{gathered}
%\end{equation}
%The probabilities of the events $A_1,A_2,\text{ and }A_3$ can be written as follows:
%\begin{equation}
%P(A_1) = P(\{X \leq x\}) = F_X(x)
%\end{equation}
%\begin{equation}
%P(A_2) = P(\{x < X \leq x + dx\}) = f_X(x)dx
%\end{equation}
%\begin{equation}
%P(A_3) = P(\{X > x + dx\}) = 1 - P(\{X \leq x + dx\})
%\end{equation}
%As $dx \rightarrow 0$, $P(\{X \leq x + dx\}) \rightarrow F_X(x)$. Therefore, $P(A_3)$ can be expressed as follows:
%\begin{equation}
%P(A_3) = 1 - F_X(x)
%\end{equation}
%Substituting values for probabilities, the following result can be generated:
%\begin{equation}
%\begin{gathered}
%f_{X_{(n)}}(x)dx=\frac{N!}{(n-1)!1!(N-n)!}\cdot\\
%[F_X(x)]^{n-1}(f_X(x)dx)[1-F_X(x)]^{N-n}
%\end{gathered}
%\end{equation}
%Dividing both sides of the equation by $dx$, the final result can be generated generated:
%\begin{equation}
%\begin{gathered}
%f_{X_{(n)}}(x)=\frac{N!}{(n-1)!(N-n)!}\cdot\\
%[F_X(x)]^{n-1}[1-F_X(x)]^{N-n}f_X(x)
%\end{gathered}
%\end{equation}

%\begin{equation}
%P_n(k_1,...,k_r)=\frac{n!}{k_1! \cdots k_r!}p_1^{k_1} \cdots p_r^{k_r}
%\end{equation}
%%\begin{equation}
%%A_1 = \{X \leq x\}
%%\end{equation}
%%\begin{equation}
%%A_2 = \{x < X \leq x + dx\}
%%\end{equation}
%%\begin{equation}
%%A_3 = \{X > x + dx\}
%%\end{equation}
%\begin{equation}
%P(A_1) = P(\{X \leq x\}) = F_X(x)
%\end{equation}
%\begin{equation}
%P(A_2) = P(\{x < X \leq x + dx\}) = f_X(x)dx
%\end{equation}
%\begin{equation}
%P(A_3) = P(\{X > x + dx\}) = 1 - P(\{X \leq x + dx\})
%\end{equation}
%%The first order statistic is the probability that one of the random variables $X$ falls in the the interval  
%As $dx \rightarrow 0$, $P(\{X \leq x + dx\}) \rightarrow F_X(x)$. Therefore, $P(A_3)$ can be expressed as follows:
%\begin{equation}
%P(A_3) = 1 - F_X(x)
%\end{equation}
%\begin{equation}
%f_{X_{(n)}}(x)dx = P\{x < X_{(n)} \leq x + dx\}
%\end{equation}
%$P\{x < X_{(n)} \leq x + dx\}$ is the probability that 1 $X$ falls in the interval $(x, x+dx]$, $n-1$ $X$'s fall in the interval $(-\infty, x]$, and $N-n$ $X$'s fall in the interval $(x+dx, \infty)$. It can be expressed in the following form:
%\begin{equation}
%\begin{gathered}
%P\{x < X_{(n)} \leq x + dx\}=\\
%\frac{N!}{(n-1)!1!(N-n)!}(P(A_1))^{n-1}P(A_2)(P(A_3))^{N-n}
%\end{gathered}
%\end{equation}
%Substituting values for probabilities, the following result can be generated:
%\begin{equation}
%\begin{gathered}
%f_{X_{(n)}}(x)dx=\frac{N!}{(n-1)!1!(N-n)!}\cdot\\
%[F_X(x)]^{n-1}(f_X(x)dx)[1-F_X(x)]^{N-n}
%\end{gathered}
%\end{equation}
%Dividing both sides of the equation by $dx$, the final result can be generated generated:
%\begin{equation}
%\begin{gathered}
%f_{X_{(n)}}(x)=\frac{N!}{(n-1)!(N-n)!}\cdot\\
%[F_X(x)]^{n-1}[1-F_X(x)]^{N-n}f_X(x)
%\end{gathered}
%\end{equation}
%
%
%The pdf of the first order statistic $X_{(1)}$ can be defined as follows:
%\begin{equation}
%f_{X_{(1)}}(x)=\frac{P(\{x < X_{(1)} \leq x+dx\})}{dx}
%\end{equation}
%After rearranging terms, the equation can be expressed as follows:
%\begin{equation}
%f_{X_{(1)}}(x)dx=P(\{x < X_{(1)} \leq x+dx\})
%\end{equation}
%If $X_k$ is the smallest random variable 
%\begin{equation}
%P(\{x < X_k \leq x+dx\} \cap \{X_i > x+dx\}) \quad \forall i \neq k 
%\end{equation}
%%\section{N-th order Statistic}
\section{Conclusion}
In this document, a set of iid random variables $\{X_1,...,X_n\}$ with pdf $f_X(x)$ was considered. This set of random variables was arranged in increasing order to produce the ordered statistics, denoted by $\{X_{(1)},...,X_{(n)}\}$. The pdf of the first-order statistic, $f_{X_{(1)}}(x)$, was derived. Next, the pdf of the Nth-order statistic, $f_{X_{(N)}}(x)$, was derived. Finally, the pdf of the overall order statistics was derived ($f_{X_{(n)}}$ for $n\in[1,N]$). These results are very useful and can be applied to many engineering problems.
%The first-order statistic $X_{(1)}$, Nth-order statistic $X_{(N)}$, and arbitrary nth-order statistic $X_{(n)}$ were considered. The pdf of the first ordered statistic $f_{X_{(1)}}(x)$ and the Nth-ordered statistic $f_{X_{(N)}}(x)$ were derived. 
%Finally, the result was generalized to an arbt
%\raggedbottom
\end{document}
